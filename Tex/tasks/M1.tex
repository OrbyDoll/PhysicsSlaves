\section{\textbf{Задача М1}. Бросок камня под углом к горизонту.}
\subsection*{Условие задачи}
Камень бросают под углом к горизонту в земном поле
тяжести с некоторой начальной скоростью. Составьте программу, которая
численно решала бы дифференциальное уравнение движения камня,
рассчитывая его траекторию движения и определяла бы точку падения.
Исследуйте, как изменяется характер траектории при различных начальных
параметрах броска (угол, начальная скорость, коэффициент сопротивления).
Для силы сопротивления воздуха рассмотрите две модели: а) вязкое трение,
пропорциональное скорости ($F \sim v$), б) лобовое сопротивление,
пропорциональное квадрату скорости ($F \sim v^2$). Там, где это возможно,
сравните результаты расчётов с теоретическими.

\subsection*{Решение}
Распишем второй закон Ньютона в векторном виде: $m\vec{\dot{v}} = m\vec{g} - k\vec{v}$. Спроецируем на ось $x$ и $y$:
\[
\begin{cases}
m\dot{v}_x = 0 - kv_x \\[14pt]
m\dot{v}_y = -mg - kv_y
\end{cases}
\]
Введем обозначение $\beta = \frac{k}{m}$. Поделим оба уравнения на массу $m$. Далее решим эти уравнения:

\[
\begin{cases}
dv_x = \beta v_x \cdot dt \\[14pt]
\dfrac{dv_y}{dt} = -g - \beta v_y
\end{cases}
\imply
\begin{cases}
\mathlarger{\int} \dfrac{dv_x}{v_x} = -\int \beta \cdot dt \\[14pt]
\mathlarger{\int} \dfrac{dv_y}{-g - \beta v_y} = \int dt
\end{cases}
\imply
\]

\[
\begin{cases}
\ln{|v_x|} = -\beta t + C \\[14pt]
\dfrac{-1}{\beta} \ln{|g+\beta v_y|} = t + C
\end{cases}
\]
\newpage
\noindent Получаем уравнения для скорости: 

\[
\begin{cases}
v_x = v_{0x} \cdot e^{-\beta t} \\[14pt]
v_y = \dfrac{\beta \cdot v_{0y} + g}{\beta} \cdot e^{-\beta t} - \dfrac{g}{\beta}
\end{cases}
\]\\
Найдем уравнения изменения координат: 

\[
\begin{cases}
\dfrac{dx}{dt} = v_{0x} \cdot e^{-\beta t} \\[14pt]
\dfrac{dy}{dt} = \dfrac{\beta \cdot v_{0y} + g}{\beta} \cdot e^{-\beta t} - \dfrac{g}{\beta}
\end{cases}
\imply
\begin{cases}
x = -\dfrac{v_{0x}}{\beta} \cdot e^{-\beta t} + C\\[14pt]
y = \dfrac{\beta \cdot v_{0y} + g}{\beta} \cdot e^{-\beta t} - \dfrac{g}{\beta}t + C
\end{cases}
\]
\[
\begin{cases}
x = -\dfrac{v_{0x}}{\beta} \cdot e^{-\beta t} + C\\[14pt]
y = \dfrac{\beta \cdot v_{0y} + g}{\beta} \cdot e^{-\beta t} - \dfrac{g}{\beta}t + C
\end{cases}
\imply
\begin{cases}
x = (1-e^{-\beta t})\cdot \dfrac{v_{0x}}{\beta} + x_0\\[14pt]
y = (1-e^{-\beta t}) \cdot \dfrac{\beta \cdot v_{0y} + g}{\beta^2} - \\ \; \; \; \; \; \; \; - \dfrac{g}{\beta}t + y_0
\end{cases}
\] \\
Итоговые уравнения зависимости координат от времени:
\[
\begin{cases}
x = (1-e^{-\beta t})\cdot \dfrac{v_0 \cos{\alpha}}{\beta} + x_0\\[14pt]
y = (1-e^{-\beta t}) \cdot \dfrac{\beta \cdot v_0 \sin{\alpha} + g}{\beta^2} - \dfrac{g}{\beta}t + y_0
\end{cases}
\]
\subsection*{Сравнение аналитического решения с результатами моделирования} 
Далее приведена таблица, в которой сравниваются показатели аналитического решения задачи с показателями этой же задачи смоделированной на компьютере используя библиотеки Python. В таблице приведены 7 экспериментов включая эксперимент №0 без сопротивления среды.

В экспериментах №5-6 показан запуск стального ядра массой 1 кг с различными видами коэффициентов сопротивления. Коэффициент сопротивления формы шара равен вычисляется по формуле: $k_{\text{лоб} }= \dfrac{C_f \rho_a S}{2} = $ \\ $= \dfrac{0.47 \cdot 1.225 \cdot 0.18}{2} \approx 0.05$. Для наглядности сравнения возьмем коэффициент вязкого трения таким же, как и лобовое.
\newpage
\begin{table}[h]
\centering
\footnotesize
\caption{Сравнение теоретических и смоделированных показателей полета}
\begin{tabular}{|c|c|c|c|c|c|c|c|c|c|}
\hline
№ & Масса & $\alpha$ & $v$ & \multirow{2}{*}{$k_{\text{вяз}}$} & \multirow{2}{*}{$k_{\text{лоб}}$} & \multicolumn{2}{c|}{Теор. показатели} & \multicolumn{2}{c|}{Моделирование} \\
\cline{7-10}
   & (кг) & (градусы) & (м/с) & & & Длина (м) & Высота (м) & Длина (м) & Высота (м) \\
\hline
Опыт №0 & 1 & 45.0 & 100 & 0 & 0 & 1020.3 & 225.1 & 1020.5 & 255.1 \\
Опыт №1 & 1 & 45.0 & 100 & 0.5 & 0 & 139.9 & 81.5 & 139.9 & 81.5 \\
Опыт №2 & 1 & 45.0 & 100 & 0.2 & 0 & 312.7 & 134.7 & 312.9 & 134.7 \\
Опыт №3 & 1 & 22.5 & 100 & 0.2 & 0 & 333.5 & 49.9 & 335.1 & 49.3 \\
Опыт №4 & 1 & 22.5 & 50 & 0.2 & 0 & 116.0 & 14.9 & 116.3 & 14.9 \\
\hline
Опыт №5 & 1 & 30 & 100 & 0.05 & 0 & 651.9 & 109.3 & 652.8 & 109.3 \\
Опыт №6 & 1 & 30 & 100 & 0  & 0.05 & - & - & 50.8 & 15.4 \\
\hline
\end{tabular}
\end{table}
\noindent \textbf{Вывод:} Данный эксперимент показал зависимость траектории полета камня от начальных условий. Значения полученные с использованием модели практически не отличаются от значений полученных аналитическим методом.